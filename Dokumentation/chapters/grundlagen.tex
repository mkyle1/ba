\chapter{Grundlagen}

Im folgenden Kapitel sollen konzeptionelle Grundlagen erläutert werden, welche für das Verständnis der Bachelorarbeit notwendig sind.
Dabei wird auf die Themen Requirements Engineering, Augmented Reality und die Software reQlab eingegangen.

  \section{Requirements Engineering}
  Die Bachelorarbeit soll sogenannte Requirements, also Anforderungen, visualisieren.
  Daher ist es für das Verständnis der Arbeit wichtig, die Grundlagen des Requirements Engineering zu kennen.

    \titleemph{Requirements}

    Grundlegend sind Requirements Anforderungen, die an ein System gestellt werden.
    Das International Requirements Engineering Board (IREB) definiert sie in ihrem Glossar mit drei Eigenschaften:
    \begin{itemize}
        \item Ein Bedürfnis eines Interesseneigners (Stakeholder).
        \item Eine Eigenschaft oder Fähigkeit, die ein System haben soll.
        \item Eine dokumentierte Repräsentation eines Bedürfnisses, einer Fähigkeit oder einer Eigenschaft.
    \end{itemize}
    \autocite[][Def. Anforderung]{ireb_cpre_glossary}

    Sie sollen also die Bedürfnisse der Stakeholder an das System repräsentieren und dokumentieren.

    Die Gestaltung von Requirements kann dabei je nach System und Anforderungen unterschiedlich sein. Chris Rupp nennt in ihrem Buch \glqq{}Requirements-Engineering und -Management\grqq{} einige Beispiele für verschiedene Formen für Requirements:
    \begin{itemize}
        \item User-Stories
        \item Use-Cases
        \item Stories
        \item formalisierte natürlichsprachliche Anforderungen
        \item Anforderungen in Form von Diagrammen (semiformales Modell)
    \end{itemize}
    \autocite[][S. 19]{Rupp2014}

    Zudem werden Requirements in funktionale und nicht-funktionale Requirements unterteilt.
    Funktionale Requirements beschreiben \glqq{}die Funktionen, die das System leisten soll, die Informationen die es verarbeiten soll; das gewünschte Verhalten, welches das System an den Tag legen soll.\grqq{} \autocite[][S. 12]{Hruschka2023}
    Nicht-funktionale Requirements hingegen beschreiben alle Requirements, die nicht funktionaler Natur sind, also beispielsweise Performance, Sicherheit oder Zuverlässigkeit.
    Peter Hruschka beschreibt in seinem Buch Funktionale Anforderungen mit der Frage: \glqq{}Was soll das System/Produkt tun?\grqq{}.
    Zudem unterteilt er nicht-funktionale Anforderungen in die zwei Kategorien Qualitätsanforderungen (\glqq{}Wie gut? Wie schnell? Wie zuverlässig? ...\grqq{}) und Randbedingungen (\glqq{}Ressourcen, Wiederverwendung, Zukauf, geforderte Technologie ...\grqq{}) \grqq{} \autocite[][S. 13]{Hruschka2023}
    Im Umfang dieser Bachelorarbeit werden jedoch nur natürlichsprachliche Anforderungen genutzt, da diese am weitesten verbreitet sind und die Anforderungen in reQlab ebenfalls in natürlicher Sprache verfasst sind.

    \titleemph{Stakeholder}

    Stakeholder können \glqq{}Personen oder Organisationen sein, die die Anforderungen eines Systems beeinflussen oder die von dem System beeinflusst werden.\grqq{} \autocite[][]{ireb_cpre_glossary}.
    Beispielsweise wären die Endnutzer eines Systems Stakeholder, welche durch das System beeinflusst werden.
    Sie haben also ein Bedürfnis an das System, können dieses jedoch nicht selbst umsetzen.
    Im Gegensatz dazu stehen die Auftraggeber beziehungsweise der Produkteigner (Product Owner), welche das System entwickeln und die Anforderungen festlegen.

    Je nach der größe und Komplexität des Systems kann es sehr viele Anforderungen geben.
    Durch die Menge an Anforderungen kann so schnell die Übersicht über das System und dessen Requirements verloren gehen, vor allem für Stakeholder, die nicht tagtäglich mit dem System arbeiten.

    \titleemph{System}

    Die IREB definiert ein System als \glqq{}Eine kohärente, abgrenzbare Menge von Elementen, die durch koordiniertes Handeln einen bestimmten Zweck erfüllen.\grqq{} \autocite[][]{ireb_cpre_glossary}
    System ist dabei ein Überbegriff für Produkte, Services, Geräte, Prozeduren und Werkzeuge und kann sowohl physisch als auch virtuell sein.

    \titleemph{Requirements Engineering}

    Requirements-Engineering ist der Prozess, in dem Anforderungen an ein System erhoben, dokumentiert, analysiert, spezifiziert und validiert werden.
    Laut Chris Rupp besteht Requirements-Engineering dabei aus vier Haupttätigkeiten:
    \begin{itemize}
        \item Wissen vermitteln
        \item Gute Anforderungen herleiten
        \item Anforderungen vermitteln
        \item Anforderungen verwalten
    \end{itemize}
    \autocite[][S.20]{Rupp2014}

  \section{reQlab}

  \section{Augmented Reality}

    \subsection{Anzeigegeräte für Augmented Reality}

    \subsubsection{Head-Mounted Displays}

    \subsubsection{Smartphones}

    \subsubsection{AR-Brillen}