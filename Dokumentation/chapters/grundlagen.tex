\chapter{Grundlagen}

In diesem Kapitel werden die Technologien vorgestellt, die im Rahmen des Praxissemesters verwendet wurden.
Dadurch soll die Grundlage zum Verständnis der Implementierung des Nutzer- und Gruppensystems in Winslow geschaffen werden.

\section{Playwright}

Die Open-Source-Bibliothek Playwright wurde Anfang 2020 von Microsoft veröffentlicht.
Die Bibliothek ermöglicht es, Browser automatisiert zu steuern und dadurch automatisierte Tests für Webanwendungen durchzuführen oder Websites zu scrapen.
Dabei bietet Playwright eine API für die Programmiersprachen JavaScript / TypeScript, Python, C\# und Java sowie eine Vielzahl von Funktionen, die das Testen von Webanwendungen erleichtern.
Beispielsweise kann die eigene Interaktion mit einer Webanwendung augezeichnet und als Code exportiert werden, der dann als Test für die ausgeführte Interaktion verwendet werden kann.

Im Fall der Studienarbeit wurde Playwright verwendet, um automatisierte Tests für eine der recherchierten Groupware-Systeme durchzuführen.
Dabei werden Frontend-Tests implementiert, die typische Interaktionen mit der Benutzeroberfläche simulieren.
So können beispielsweise Formulare ausgefüllt oder Buttons angeklickt werden.

Deckt man mit diesen Tests alle Funktionsbereiche des Groupwaresystems ab, kann man durch das Ausführen der Tests sicherstellen, dass die Anwendung nach einer Änderung noch wie erwartet funktioniert.
Auch falls die Anwendung in Zukunft unerwartete Ausfälle generiert, können diese durch die Tests schnell erkannnt werden.





Dabei wird die Anwendung in einem simulierten Browser geöffntet und die Interaktionen werden durchgeführt.

\autocite[Quelle:][]{angular}





