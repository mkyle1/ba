% Eigenkorrekturlesung 1 fertig
\chapter{Grundlagen}

Dieses Kapitel soll technologische und konzeptionelle Grundlagen für das Verständnis der Studienarbeit liefern.
Zuerst werden anhand eines Beispiels Groupware-Systeme und deren grundlegende Funktionalitäten dargestellt, um ein Verständnis für die Funktionen von Groupware-Systemen zu schaffen.
Daraufhin wird die Open-Source-Bibliothek Playwright vorgestellt, die in dieser Studienarbeit für die Implementierung von automatisierten End-to-End-Tests verwendet wurde.
Dabei wird auf die Gründe und Vorteile von End-to-End-Tests, sowie die Funktionsweise von Playwright eingegangen.

\section{Groupware-Systeme}

Groupware-Systeme sind Software-Systeme, welche die Zusammenarbeit von Benutzern mit verschiedenen Tools zur gemeinsamen Kommunikation und Organisation unterstützen.
Dabei werden üblicherweise Funktionalitäten wie Kalender, Terminplanung, E-Mails und Kontaktmanagement bereitgestellt.

Im Folgenden werden anhand von Screenshots aus Microsoft Outlook Live einige essenzielle Funktionen von Groupware-Systemen dargestellt und erklärt.
\begin{figure}[H]
    \centering
    \includegraphics[width=0.75\textwidth]{images/OutlookLive_Mail1.png}
    \caption{Screenshot von Microsoft Outlook Live Mail}
    \source{https://outlook.live.com/calendar/0/view/month, abgerufen am 29.01.2024}
    \label{fig:outlook-live-mail}
\end{figure}
Die erste Hauptfunktion von Groupware-Systemen besteht darin, einen E-Mail-Client bereitzustellen, über den E-Mails empfangen, versendet und verwaltet werden können. Dies wird beispielhaft anhand von Outlook Live in Abbildung \ref{fig:outlook-live-mail} dargestellt.
Zudem sollten alle Funktionen eines modernen E-Mail-Clients, wie beispielsweise das Sortieren von E-Mails in Ordnern oder das Hinzufügen mehrerer E-Mail-Postfächer gleichzeitig.


\begin{figure}[H]
    \centering
    \includegraphics[width=0.75\textwidth]{images/OutlookLive_Calender1.png}
    \caption{Screenshot von Microsoft Outlook Live Kalender}
    \source{https://outlook.live.com/mail/0/, abgerufen am 29.01.2024}
    \date{29.01.2024}
    \label{fig:outlook-live-calender}
\end{figure}

Eine weitere Hauptfunktion von Groupware-Systemen ist ein Kalender, der Nutzern eine Terminplanung mit Ereignissen ermöglicht (siehe Abbildung \ref{fig:outlook-live-calender}).
Die Terminplanung sollte es den Nutzern ermöglichen, andere Nutzer aus den Kontakten einzuladen, um die Zusammenarbeit und gemeinsame Organisation zu vereinfachen.
Auch das Anlegen von Regelterminen, das heißt Terminen, die in regelmäßigen Zeitabständen wiederkehrend sind, sollte möglich sein.

\begin{figure}[H]
    \centering
    \includegraphics[width=0.75\textwidth]{images/OutlookLive_Contacts.png}
    \caption{Screenshot von Microsoft Outlook Live Kontakte}
    \source{https://outlook.live.com/people/0/, abgerufen am 29.01.2024}
    \label{fig:outlook-live-contacts}
\end{figure}

Kontakte sind ein weiterer wichtiger Bestandteil von Groupware-Systemen, um die Vernetzung innerhalb von Arbeitsgruppen zu organisieren.
Sie ermöglichen eine einfache Kontaktaufnahme zu anderen Gruppenmitgliedern.
In Outlook-Live kann man, wie in Abbildung \ref{fig:outlook-live-contacts} gezeigt, direkt vom Kontakt einer Person diese Person per Nachricht, Anruf oder E-Mail kontaktieren.


\section{Playwright}

Die Open-Source-Bibliothek Playwright, die Anfang 2020 von Microsoft veröffentlicht wurde, ermöglicht die gesteuerte Automatisierung von Browsern und Webinterfaces. Dadurch ist es möglich, Webanwendungen automatisiert zu testen oder Websites abzutasten.
Dabei bietet Playwright ein Application-Programming-Interface (API) für die Programmiersprachen JavaScript, TypeScript, Python, .NET und Java, sowie eine Vielzahl von Funktionen, die das Testen von Webanwendungen erleichtern.
Beispielsweise kann mit Playwright-Codegen die eigene Interaktion mit einer Webanwendung aufgezeichnet und als Code exportiert werden, der dann als Test für die ausgeführte Interaktion verwendet werden kann.
So können effizient Frontend-Tests für eine Vielzahl von Anwendungen implementiert werden \autocite[][]{playwright}.

Im Fall der Studienarbeit wurde Playwright verwendet, um automatisierte End-to-End-Tests für das final ausgewählte Groupware-System durchzuführen.
End-to-End-Tests sind Tests, die reale Nutzerszenarien simulieren, um sicherzustellen, dass die Anwendung wie gewünscht funktioniert.
Die Tests sind dabei so konzipiert, dass sie die Anwendung aus Sicht eines Endnutzers testen und so das korrekte Betriebsverhalten aus einer realen Umgebung sicherstellen.
So können sie, vor allem bei Anwendungen, die viele Nutzerinteraktionen erfordern, die User-Experience verbessern \autocite[vgl.][]{e2e-blog}.
Dafür werden Frontend-Tests implementiert, die typische Interaktionen mit der Benutzeroberfläche simulieren.
Beispielsweise können so Formulare ausgefüllt oder Buttons angeklickt werden, womit beispielsweise ein Nutzer-Login und das anschließende Aufrufen der Mails des Nutzers simuliert werden kann.

Deckt man mit diesen Tests alle Funktionsbereiche des Groupware-Systems ab, kann man durch das Ausführen der Tests sicherstellen, dass die vollständige Anwendung nach einer Änderung noch wie erwartet funktioniert.
Auch falls die Anwendung in Zukunft unerwartete Ausfälle generiert, können diese durch flächendeckende Tests genauer erkannt werden, da sofort ersichtlich ist, welche Bereiche des Systems noch funktionieren und welche nicht.
Geht beispielsweise der zuvor erwähnte Test des aufrufen der Mails schief, gibt es mit hoher Wahrscheinlichkeit ein Problem mit der Verbindung zum Mail Server.

Zudem können solche Tests in Playwright in verschiedenen Browsern (Chromium, Firefox, WebKit) ausgeführt werden, wodurch die Funktionalität der Anwendung auch auf verschiedenen Browsern sichergestellt und kontinuierlich getestet werden kann.
Da die Groupware von einer Vielzahl an Systemen zugänglich sein soll, ist diese Funktion ein essenzieller Bestandteil der Testanforderungen.

% Word korrigiert






