\chapter{Einleitung}

In diesem Kapitel werden die Hintergründe der Studienarbeit zur Recherche, Bewertung, Implementierung und dem automatisierten Testen von Groupware-Systemen erläutert.

\section{Grund für die Studienarbeit}

Die Hochschule Esslingen setzt momentan Microsoft Exchange als Groupware-System ein.
Aufgrund der Lizenzkosten und der Abhängigkeit von Microsoft möchte die Hochschule Esslingen ein Open Source Groupware-System evaluieren und testen.


\section{Ziele der Studienarbeit}

Im Laufe der Studienarbeit sollen verschiedene Groupware-Systeme recherchiert, bewertet und getestet werden.
Davon soll mindestens ein Groupware-System anhand der in Kapitel 2.3 dargestellten Kriterien ausgewählt und anschließend in einer Testumgebung installiert und konfiguriert werden.
Dieses System soll dann automatisiert mit Playwright getestet werden.
Die Playwright-Tests sollen übliche Anwendungsfälle von Nutzern und Administratoren abdecken und eine erste Einschätzung über die Qualität bzw. Stabilität des Systems geben.



\section{Kriterien für das Groupware-System}
\begin{itemize}
    \item \textbf{Open Source:} 
    Das erste und wichtigste Kriterium ist, dass das Groupware-System Open Source ist.
    Daher werden im Laufe der Recherche nur Open Source Groupware-Systeme betrachtet.
    \item \textbf{Eigenverwaltbarkeit:}
    Die Hochschule Esslingen hat ein eigenes Rechnerzentrum und eine IT-Fakultät.
    Daher sollte die Software von der Hochschule Esslingen selbst installiert und administriert werden können.
    \item \textbf{Deutsche Firma:}
    Als deutsche Hochschule möchte die Hochschule Esslingen auch deutsche Firmen unterstützen.
    Deshalb ist es eine Vorgabe, dass das Groupwaresystem von einer deutschen Firma entwickelt wird.
    Dies ist zwar ein wichtiges Kriterium, muss aber nicht zwingend zum Ausschluss führen.
    
\end{itemize}
