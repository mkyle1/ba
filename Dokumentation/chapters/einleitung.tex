\chapter{Einleitung}

In diesem Kapitel werden die Hintergründe der Studienarbeit zur Recherche, Bewertung, Implementierung und dem automatisierten Testen von Groupwaresystemen erläutert.

\section{Grund für die Suche eines neuen Groupwaresystems}

\section{Kriterien für das Groupwaresystem}%DeepL korrigiert
\begin{itemize}
    \item \textbf{Open Source}: 
    Die Open Source Lizenz des Groupwaresystems ist eine Vorgabe, da die Software von der Hochschule Esslingen verwendet werden soll.
    Das ist dabei ein Ausschlusskriterium, was bedeutet, dass nur Groupwaresysteme, die eine Open Source Lizenz besitzen, überhaupt in Frage kommen.
    \item \textbf{Deutsche Firma}:
    Als deutsche Hochschule möchte die Hochschule Esslingen auch deutsche Firmen unterstützen.
    Deshalb ist es eine Vorgabe, dass das Groupwaresystem von einer deutschen Firma entwickelt wird.
    Dies ist zwar ein wichtiges Kriterium, muss aber nicht zwingend zum Ausschluss führen.
    \item \textbf{Eigenverwaltbarkeit}:
    Die Hochschule Esslingen hat ein eigenes Rechnerzentrum und eine IT-Fakultät.
    Daher sollte die Software von der Hochschule Esslingen selbst administriert werden können.
\end{itemize}
