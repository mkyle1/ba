\chapter{Einleitung}

\section{Motivation}

\section{Zielsetzung}

Im Laufe der Bachelorarbeit sollen verschiedene Interaktionskonzepte für die Anzeige von Anforderungen in einer AR-Umgebung untersucht und auf ihre Vor- und Nachteile, sowie auf ihre Eignung für den Einsatz in einem realen Projekt untersucht werden.

Dabei sollen möglichst mehrere Konzepte entwickelt und prototypisch umgesetzt werden, um diese anschließend zu evaluieren und zu vergleichen.
Der Fokus der Evaluation soll auf der Usability und der Umsetzbarkeit der Konzepte liegen.
Vor allem der Mehrwert der Darstellung in AR gegenüber herkömmlichen Methoden soll dabei kritisch betrachtet werden.

Basierend auf dem Prozess der Implementierung soll auch bewertet werden, wie gut sich die Konzepte in bestehende Systeme integrieren lassen und wie aufwändig die Implementierung im Vergleich zum Mehrwert der Darstellung ist.
Besonders interessant ist dabei, inwiefern eine Integration mit der reQlab Plattform möglich ist und wie sich die Konzepte in bestehende Workflows einfügen lassen.