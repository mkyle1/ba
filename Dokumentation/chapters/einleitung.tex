\chapter{Einleitung}

In diesem Kapitel wird eine thematische Einleitung in die Motivation und die Ziele der Arbeit gegeben.
Im Laufe der Arbeit wird der Begriff Requirements-Engineering anstatt des deutschen Begriffs Anforderungsanalyse und Requirements anstatt Anforderungen verwendet.

\section{Motivation}
\label{section:motivation}

Das Requirements-Engineering ist ein wichtiger Bestandteil der Softwareentwicklung und dient dazu, die Requirements an ein System zu erfassen und zu spezifizieren.
Gut definierte Requirements sollen genau beschreiben, was von einem System erwünscht ist und welche Funktionalitäten es bieten soll.
Sind die Requirements von Systemen nicht genau definiert, kann es zu Missverständnissen und Fehlinterpretationen in der Entwicklung und in schlimmen Fällen bis in das Endprodukt kommen.
Da diese Fehler meist erst spät im Entwicklungsprozess erkannt werden, sind sie oft mit hohen Kosten verbunden.
Daher wird in der Industrie viel Wert auf ein genaues und strukturiertes Requirements-Engineering gelegt.
Ein höherer Aufwand im Requirements-Engineering kann dabei helfen, Fehler frühzeitig zu erkennen und zu vermeiden und so Kosten durch Fehler und Missverständnisse zu reduzieren.

Deshalb ist es wichtig, dass die Requirements klar und verständlich formuliert sind, um Missverständnisse und Fehlinterpretationen zu vermeiden.
Einen Ansatz zur Verbesserung der Qualität der Requirements stellt reQlab dar, ein Projekt der IT-Designers GmbH in dessen Rahmen diese Bachelorarbeit entstanden ist.
Das Tool unterstützt Nutzer bei der korrekten Formulierung und Strukturierung von Requirements.
reQlab untersucht dafür die einzelnen Requirements auf ihre Qualität und gibt eine Bewertung ab, wie gut sie formuliert sind.
Dadurch kann das Tool die Qualität der einzelnen Requirements stark verbessern und so die Qualität des gesamten Systems steigern.

Der durch reQlab verfolgte Ansatz zielt jedoch nur auf die Verbesserung eines Teils des Requirements-Engineering.
Ein weiterer wichtiger Aspekt ist die kontinuierliche Kommunikation der Requirements zwischen Entwicklern und Auftraggebern.
Da es, egal wie viel Aufwand in das Requirements-Engineering gesteckt wird, immer zu Missverständnissen und Fehlinterpretationen kommen kann, ist es wichtig, dass die Requirements regelmäßig überprüft und diskutiert werden.
Dabei sollte darauf geachtet werden, dass alle Beteiligten möglichst alle Requirements die sie betreffen kennen, verstehen und nachvollziehen können.
Jedoch ist es oft schwer, vor allem als Auftraggeber, der nicht täglich in den Entwicklungsprozess involviert ist, einen vollen Überblick über alle relevanten Requirements zu behalten und zu verstehen, was genau gefordert wird.

Gleichzeitig öffnen sich mit dem Fortschritt der Technik immer mehr Möglichkeiten, um Requirements zu visualisieren und zu präsentieren.
Augmented Reality Technologien werden durch neue Endgeräte wie die Meta Quest 3 oder die Microsoft HoloLens immer zugänglicher und bieten neue Möglichkeiten, um Daten zu visualisieren und zu präsentieren.

Daher soll in dieser Bachelorarbeit untersucht werden, ob sich Requirements an Systeme in einer AR-Umgebung darstellen lassen und ob dadurch ein Mehrwert gegenüber herkömmlichen Darstellungsmethoden entstehen kann.


\section{Zielsetzung}

Im Laufe der Bachelorarbeit sollen verschiedene Interaktionskonzepte für die Anzeige von Requirements in einer AR-Umgebung untersucht und auf ihre Vor- und Nachteile, sowie auf ihre Eignung für den Einsatz in einem realen Projekt untersucht werden.

Dabei sollen möglichst mehrere Konzepte entwickelt und prototypisch umgesetzt werden, um diese anschließend zu evaluieren und zu vergleichen.
Die Konzepte sollen dabei möglichst unterschiedliche Ansätze der Darstellung und Interaktion verfolgen, um so die Vor- und Nachteile der verschiedenen Interaktionskonzepte zu untersuchen.

Die Prototypen sollen in der Lage sein die Interaktionskonzepte anhand von Beispielen darzustellen.
Sie müssen dabei nicht jedes Feature der ausgearbeiteten Interaktionskonzepte umsetzen, sondern sollen vor allem die Interaktionsmöglichkeiten und den Mehrwert der Darstellung in AR gegenüber herkömmlichen Methoden zeigen.
Daher soll jeder Prototyp mindestens zwei verschiedene Interaktionsmöglichkeiten mit den Anforderungen bieten, um einen hohen Grad an Interaktivität zu gewährleisten und gleichzeitig den Aufwand der Implementierung der Prototypen angemessen zu halten.
Während der Entwicklung soll dabei untersucht werden, wie einfach die Konzepte technisch umsetzbar sind, um eine Einschätzung des Aufwands einer echten Implementierung zu erlangen.
Ein wichtiger Faktor bei der Umsetzbarkeit spielt dabei, ob die Umsetzung automatisiert werden könnte, um realistisch in einer realen Anwendung eingesetzt zu werden.
Konzepte, die viel Handarbeit und somit viel Zeit und Geld für die Implementierung benötigen, müssen auch einen dementsprechend höheren Mehrwert bieten, um die tatsächliche Umsetzung zu rechtfertigen. 

\newpage

Der Fokus der Evaluation der Prototypen soll dann vor allem auf den folgenden Kriterien liegen:

\begin{itemize}
    \item Usability: Wie einfach und intuitiv ist die Bedienung der Prototypen?
    \begin{itemize}
        \item Mehrwert: Bieten die Prototypen einen Mehrwert der Usability?
        \item AR: Ist die Darstellung in AR sinnvoll und bietet sie einen Mehrwert gegenüber herkömmlicher Methoden? Wären die Interaktionskonzepte auch ohne AR sinnvoll?
    \end{itemize}
    \item Umsetzbarkeit: Wie aufwändig ist die Implementierung der Prototypen und wie gut lassen sie sich in bestehende Systeme integrieren?
    \begin{itemize}
        \item Automatisierbarkeit: Wie einfach könnte die Umsetzung der Prototypen automatisiert werden?
        \item Aufwand: Wie hoch wäre der Aufwand für die Realisierung der Interaktionskonzepte in einer realen Anwendung?
    \end{itemize}
\end{itemize}

Die beiden Kriterien und ihre Unterkriterien sollen dabei helfen, die Vor- und Nachteile der verschiedenen Konzepte zu identifizieren und zu bewerten.
Anhand der Ergebnisse der Evaluation soll dann eine Einschätzung gegeben werden, ob sich die Konzepte für die Implementierung in einer realen Anwendung für reale Projekte eignen.
Dabei soll der entstehende Mehrwert der Darstellung gegenüber herkömmlichen Methoden im Gegenzug zum Aufwand der Implementierung bewertet werden.

% 1x Durchgelesen (02.07)
