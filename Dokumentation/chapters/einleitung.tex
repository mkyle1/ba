\chapter{Einleitung}

In diesem Kapitel werden die Hintergründe der Studienarbeit zur Recherche, Bewertung, Implementierung und dem automatisierten Testen von Groupware-Systemen erläutert.

\section{Motivation für die Studienarbeit}

Die Hochschule Esslingen setzt momentan Microsoft Exchange als Groupware-System ein.
Aufgrund der Lizenzkosten und der Abhängigkeit von Microsoft möchte die Hochschule Esslingen ein neues Groupware-System einsetzen.
Um bei der Entscheidung zu helfen, soll in dieser Studienarbeit zumindest ein Groupware-System auf seine Eignung für die Hochschule Esslingen getestet werden.
Dabei soll nicht bewiesen werden, dass das ausgewählte System das beste der Open-Source-Groupware-Systeme ist, sondern ob es für die Hochschule Esslingen geeignet ist.

Aufbauend auf diese Studienarbeit sollen dann eventuell weitere Studienarbeiten folgen, die andere Groupware-Systeme testen und bewerten.
So soll der Hochschule Esslingen eine fundierte Entscheidungshilfe für die Auswahl eines neuen Groupware-Systems gegeben werden.



\section{Ziele der Studienarbeit}

Im Laufe der Studienarbeit sollen verschiedene Open-Source Groupware-Systeme recherchiert, bewertet und getestet werden.
Davon soll mindestens ein Groupware-System anhand später erläuterter Kriterien ausgewählt und anschließend in einer Testumgebung installiert und konfiguriert werden.
Dieses System soll dann automatisiert mit der User-Interface-Testing-Bibliothek Playwright getestet werden.
Die Playwright-Tests sollen übliche Anwendungsfälle von Nutzern und Administratoren abdecken und eine erste Einschätzung über die Qualität bzw. Stabilität des Systems geben.



