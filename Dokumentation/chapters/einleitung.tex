% Eigenkorrekturlesung 1 fertig

\chapter{Einleitung}

In diesem Kapitel wird auf die Motivation für die Suche eines neuen Groupware-Systems für die Hochschule Esslingen eingegangen und die Ziele der genaueren Untersuchung eines der Groupware-Systeme erläutert.


\section{Motivation für die Studienarbeit}

Die Hochschule Esslingen setzt momentan Microsoft Exchange als Groupware-System ein.
Aufgrund der Lizenzkosten und der Abhängigkeit von Microsoft möchte die Hochschule Esslingen ein neues Groupware-System einsetzen.
Um bei der Entscheidung zu helfen, soll diese Studienarbeit einen ersten Überblick über verfügbare Groupware-Systeme geben und mindestens ein Groupware-System auf seine Eignung für die Hochschule Esslingen getestet werden.

Aufbauend auf diese Studienarbeit sollen dann eventuell weitere Studienarbeiten folgen, die andere Groupware-Systeme testen und bewerten.
So soll der Hochschule Esslingen eine fundierte Entscheidungshilfe für die Auswahl eines neuen Groupware-Systems gegeben werden.


\section{Ziele der Studienarbeit}

Im Laufe der Studienarbeit sollen verschiedene Open-Source-Groupware-Systeme recherchiert und basierend auf ihre Eignung als Groupware-System für die Hochschule Esslingen bewertet werden.
Davon soll mindestens ein Groupware-System anhand später erläuterter Kriterien ausgewählt und anschließend in einer Testumgebung installiert und grundlegend konfiguriert werden.
Dieses System soll dann automatisiert mit der User-Interface-Testing-Bibliothek Playwright getestet werden.
Die Playwright-Tests sollen übliche Anwendungsfälle von Nutzern und Administratoren abdecken und eine erste Einschätzung über die Qualität bzw. Stabilität des Systems geben.

Über das getestete Groupware-System soll dann anhand der Tests ein Fazit gezogen werden, ob es für die Anforderungen Hochschule Esslingen geeignet ist.
Dabei soll auch ein Ausblick auf mögliche nächste Schritte in der Suche nach einer Groupware-Lösung für die Hochschule Esslingen gegeben werden.



