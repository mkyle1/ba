\chapter{Einleitung} % DeepL Korrigiert, final korrigiert

In diesem Kapitel wird eine thematische Einführung in die Motivation und die Ziele der vorliegenden Arbeit gegeben.
Im Rahmen der Arbeit wird der Begriff \glqq{}Requirements-Engineering\grqq{} anstatt des deutschen Begriffs \glqq{}Anforderungsanalyse\grqq{} und \glqq{}Requirements\grqq{} anstatt \glqq{}Anforderungen\grqq{} verwendet.

\section{Motivation}
\label{section:motivation}

Das Requirements-Engineering stellt einen wesentlichen Bestandteil der Entwicklung von Systemen dar und dient der Erfassung sowie Spezifikation der Requirements an ein System.
Requirements, die gut formuliert sind, definieren die gewünschten Funktionalitäten eines Systems und beschreiben dessen Eigenschaften.
Eine unzureichende Spezifikation der Requirements eines Systems birgt das Risiko von Missverständnissen und Fehlinterpretationen in der Entwicklung, die bis in das Endprodukt hineinreichen können.
Da diese Fehler meist erst spät im Entwicklungsprozess identifiziert werden, gehen mit ihnen in der Regel hohe Kosten einher.
Daher wird in der Industrie viel Wert auf ein genaues und strukturiertes Requirements-Engineering gelegt.
Ein höherer Aufwand im Requirements-Engineering kann dazu beitragen, Fehler frühzeitig zu erkennen und zu vermeiden, wodurch letztlich Kosten, die durch Fehler und Missverständnisse entstehen, reduziert werden können.

Daher ist es von essenzieller Bedeutung, dass die Requirements klar und verständlich formuliert sind, um potenzielle Missverständnisse und Fehlinterpretationen zu vermeiden.
Einen Ansatz zur Verbesserung der Qualität der Requirements stellt reQlab dar, ein Projekt der IT-Designers GmbH in dessen Rahmen diese Bachelorarbeit entstanden ist.
Das Tool unterstützt Nutzer bei der korrekten Formulierung und Strukturierung von Requirements.
Dazu untersucht reQlab die einzelnen Requirements auf ihre Qualität und gibt eine Bewertung ab, wie gut sie formuliert sind.
Auf diese Weise kann das Tool die Qualität der einzelnen Requirements signifikant verbessern und somit auch die Qualität des gesamten Systems steigern.

Der von reQlab verfolgte Ansatz zielt jedoch lediglich auf die Optimierung eines Teils des Requirements-Engineerings ab.
Ein weiterer wesentlicher Aspekt ist die kontinuierliche Kommunikation der Requirements zwischen Entwicklern und Auftraggebern.
Da trotz eines umfangreichen Aufwands im Requirements-Engineering eine vollständige Vermeidung von Missverständnissen und Fehlinterpretationen nicht möglich ist, ist es unerlässlich, die Requirements einer regelmäßigen Überprüfung und Diskussion zu unterziehen.
Es ist von essenzieller Bedeutung, dass alle Beteiligten über alle Requirements, die sie betreffen, informiert sind, diese verstehen und nachvollziehen können.
Allerdings erweist es sich in der Praxis häufig als schwierig -- insbesondere für Auftraggeber, die nicht täglich in den Entwicklungsprozess involviert sind -- einen umfassenden Überblick über alle relevanten Requirements zu gewinnen.

Gleichzeitig eröffnet der Fortschritt der Technik eine Vielzahl neuer Möglichkeiten, um Daten zu visualisieren und zu präsentieren.
Die Verfügbarkeit neuer Endgeräte wie der Meta Quest 3 oder der Microsoft HoloLens führt zu einer zunehmenden Zugänglichkeit von Augmented-Reality-Technologien, wodurch sich potenziell neue Möglichkeiten zur Visualisierung und Präsentation von Requirements eröffnen.

In der vorliegenden Bachelorarbeit soll daher untersucht werden, ob sich Requirements in einer AR-Umgebung darstellen lassen und ob dadurch ein Mehrwert gegenüber herkömmlichen Darstellungsmethoden entsteht.

% DeepL korrigiert


\section{Zielsetzung}

Im Rahmen der Bachelorarbeit erfolgt eine Untersuchung diverser Interaktionskonzepte für die Anzeige von Requirements in einer AR-Umgebung.
Ziel ist es, die Vor- und Nachteile der einzelnen Ansätze sowie ihre Eignung für den Einsatz in einem realen Projekt zu untersuchen.

Im Rahmen der Untersuchung sollen möglichst mehrere Konzepte entwickelt und prototypisch umgesetzt werden, um diese anschließend einer Evaluierung und einem Vergleich zu unterziehen.
Bei der Entwicklung der Konzepte soll darauf geachtet werden, dass möglichst unterschiedliche Ansätze hinsichtlich Darstellung und Interaktion verfolgt werden, um auf diese Weise die jeweiligen Vor- und Nachteile der verschiedenen Konzepte untersuchen zu können.

Die zu entwickelnden Prototypen sollen in der Lage sein, die Interaktionskonzepte anhand von Beispielen zu veranschaulichen.
Es ist nicht erforderlich, sämtliche Features der ausgearbeiteten Interaktionskonzepte umzusetzen.
Vielmehr sollen die Prototypen die Interaktionsmöglichkeiten und den Mehrwert der Darstellung in AR gegenüber herkömmlichen Methoden veranschaulichen. \newline
Daher soll jeder Prototyp mindestens zwei verschiedene Interaktionsmöglichkeiten mit den Requirements bieten, um einerseits einen hohen Grad an Interaktivität zu gewährleisten und andererseits den Aufwand der Implementierung der Prototypen angemessen zu halten.
Im Rahmen der Entwicklung ist zu untersuchen, inwiefern die Konzepte technisch umsetzbar sind, um eine Einschätzung des Aufwands einer echten Implementierung zu erlangen.
Ein wesentlicher Aspekt bei der Umsetzbarkeit ist die Frage, ob eine automatisierte Umsetzung möglich ist, um eine realistische Integration in einer realen Anwendung zu gewährleisten.
Konzepte, die einen hohen manuellen Aufwand bei der Implementierung erfordern, müssen einen entsprechend höheren Mehrwert bieten, um die tatsächliche Umsetzung zu rechtfertigen. 

Bei der Evaluation der Prototypen sollen insbesondere die folgenden Kriterien berücksichtigt werden:

\begin{itemize}
    \item Usability: Wie einfach und intuitiv ist die Bedienung der Prototypen?
    \begin{itemize}
        \item Mehrwert: Bieten die Prototypen einen Mehrwert der Usability?
        \item AR vs 2D: Ist die Darstellung in AR sinnvoll und bietet sie einen Mehrwert gegenüber herkömmlicher zweidimensionaler Darstellungsmethoden? Wären die Interaktionskonzepte auch ohne AR sinnvoll?
    \end{itemize}
    \item Umsetzbarkeit: Wie aufwändig ist die Implementierung der Prototypen und wie gut lassen sie sich in bestehende Systeme integrieren?
    \begin{itemize}
        \item Automatisierbarkeit: Wie einfach könnte die Umsetzung der Prototypen automatisiert werden?
        \item Aufwand: Wie hoch wäre der Aufwand für die Realisierung der Interaktionskonzepte in einer realen Anwendung?
    \end{itemize}
\end{itemize}

Die beiden Kriterien und ihre Unterkriterien dienen der Identifizierung und Bewertung der Vor- und Nachteile der verschiedenen Konzepte.
Auf Basis der Evaluationsergebnisse erfolgt eine Einschätzung hinsichtlich der Eignung der Konzepte für die Implementierung in einer realen Anwendung für die Verwaltung echter Projekte.
Der resultierende Mehrwert der Darstellung gegenüber den herkömmlichen Methoden ist gegen den Aufwand der Implementierung aufzuwiegen.

% DeepL Korrigiert
% 1x Durchgelesen (02.07)
