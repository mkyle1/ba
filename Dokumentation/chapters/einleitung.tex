\chapter{Einleitung}

\section{Motivation}

Die Anforderungsanalyse ist ein wichtiger Bestandteil der Softwareentwicklung und dient dazu, die Anforderungen an ein System zu erfassen und zu spezifizieren.
Sie sind dazu da, um genau zu definieren, was von einem System erwünscht ist und welche Funktionalitäten es bieten soll.
Sind die Anforderungen von Systemen nicht genau definiert, kann es zu Missverständnissen und Fehlinterpretationen in der Entwicklung und in schlimmen Fällen bis in das Endprodukt kommen.
Da diese Fehler meist erst spät im Entwicklungsprozess erkannt werden, sind sie oft mit hohen Kosten verbunden.
Daher wird in der Industrie viel Wert auf eine genaue und strukturierte Anforderungsanalyse gelegt.
Ein höherer Aufwand in der Anforderungsanalyse kann dabei helfen, Fehler frühzeitig zu erkennen und zu vermeiden und so Kosten durch Fehler und Missverständnisse zu reduzieren.

Deshalb ist es wichtig, dass die Anforderungen klar und verständlich formuliert sind, um Missverständnisse und Fehlinterpretationen zu vermeiden.
Einen Ansatz zur Verbesserung der Qualität der Anforderungen stellt reQlab dar, das bei der korrekten Formulierung und Strukturierung von Anforderungen unterstützt.
reQlab untersucht die einzelnen Anforderungen auf ihre Qualität und gibt eine Bewertung ab, wie gut sie formuliert sind.
Dadurch kann reQlab die Qualität der einzelnen Anforderungen stark verbessern und so die Qualität des gesamten Systems steigern.

Der durch reQlab verfolgte Ansatz zielt jedoch nur auf die Verbesserung eines Teils der Anforderungsanalyse.
Ein weiterer wichtiger Aspekt ist die kontinuierliche Kommunikation der Anforderungen zwischen Entwicklern und Auftraggebern.
Da, egal wie viel Aufwand in die Anforderungsanalyse gesteckt wird, es immer zu Missverständnissen und Fehlinterpretationen kommen kann, ist es wichtig, dass die Anforderungen regelmäßig überprüft und diskutiert werden.
Dabei ist es wichtig, dass alle Beteiligten möglichst alle Anforderungen kennen, verstehen und nachvollziehen können.
Jedoch ist es oft schwer, vor allem als Auftraggeber, der nicht täglich in den Entwicklungsprozess involviert ist, einen vollen Überblick über alle Anforderungen zu behalten und zu verstehen, was genau gefordert wird.

Gleichzeitig öffnen sich mit dem Fortschritt der Technik immer mehr Möglichkeiten, um Anforderungen zu visualisieren und zu präsentieren.
Augmented Reality Technologien werden durch neue Endgeräte wie die Meta Quest 3 oder die Microsoft HoloLens immer zugänglicher und bieten neue Möglichkeiten, um Daten zu visualisieren und zu präsentieren.

Daher soll in dieser Bachelorarbeit untersucht werden, ob sich Anforderungen an Systeme in einer AR-Umgebung darstellen lassen und ob dadurch ein Mehrwert gegenüber herkömmlichen Darstellungsmethoden entstehen kann.


\section{Zielsetzung}

Im Laufe der Bachelorarbeit sollen verschiedene Interaktionskonzepte für die Anzeige von Anforderungen in einer AR-Umgebung untersucht und auf ihre Vor- und Nachteile, sowie auf ihre Eignung für den Einsatz in einem realen Projekt untersucht werden.

Dabei sollen möglichst mehrere Konzepte entwickelt und prototypisch umgesetzt werden, um diese anschließend zu evaluieren und zu vergleichen.
Die Konzepte sollen dabei möglichst unterschiedliche Ansätze der Darstellung und Interaktion verfolgen, um so die Vor- und Nachteile der verschiedenen Ansätze zu untersuchen.

Die Prototypen sollen in der Lage sein die Interaktionskonzepte anhand von Beispielen darzustellen.
Dabei soll jeder Prototyp mindestens 2 verschiedene Interaktionsmöglichkeiten mit den Anforderungen bieten, um einen hohen Grad an Interaktivität zu gewährleisten.
Während der Entwicklung soll dabei untersucht werden, wie einfach die Konzepte technisch umsetzbar sind und ob die Umsetzung automatisiert werden könnte, um in einer realen Anwendung eingesetzt zu werden.

Der Fokus der Evaluation der Prototypen soll dann vor allem auf den folgenden Kriterien liegen:

\begin{itemize}
    \item Usability: Wie einfach und intuitiv ist die Bedienung der Prototypen?
    \begin{itemize}
        \item Mehrwert: Bieten die Prototypen einen Mehrwert der Usability?
        \item AR: Ist die Darstellung in AR sinnvoll und bietet sie einen Mehrwert gegenüber herkömmlicher Methoden? Wären die Interaktionskonzepte auch ohne AR sinnvoll?
    \end{itemize}
    \item Umsetzbarkeit: Wie aufwändig ist die Implementierung der Prototypen und wie gut lassen sie sich in bestehende Systeme integrieren?
    \begin{itemize}
        \item Automatisierbarkeit: Wie einfach könnte die Umsetzung der Prototypen automatisiert werden?
        \item Aufwand: Wie hoch wäre der Aufwand für die Realisierung der Interaktionskonzepte in einer realen Anwendung?
    \end{itemize}
\end{itemize}

Die beiden Kriterien und ihre Unterkriterien sollen dabei helfen, die Vor- und Nachteile der verschiedenen Konzepte zu identifizieren und zu bewerten.
Anhand der Ergebnisse der Evaluation soll dann eine Enschätzung gegeben werden, ob sich die Konzepte für die Implementierung in einer realen Anwendung für reale Projekte eignen.
Dabei soll der entstehende Mehrwert der Darstellung gegenüber herkömmlichen Methoden im Gegenzug zum Aufwand der Implementierung bewertet werden.
