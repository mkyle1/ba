\chapter{Zusammenfassung}

In diesem Kapitel wird ein Fazit über en Verlauf des Studienprojekts und das im Rahmen des Studienprojekts untersuchte Groupware-System als Lösung für die Hochschule Esslingen gezogen.
Zuletzt wird ein Ausblick auf mögliche nächste Schritte in der Suche nach einer Groupware-Lösung für die Hochschule Esslingen gegeben.

\section{Fazit}

Das Ziel des Studienprojekts war es, ein Groupware-System zu finden und zu testen, das die Anforderungen der Hochschule Esslingen erfüllt.
Dabei wurden die Anforderungen an das Groupware-System in einem ersten Schritt definiert und anschließend die verfügbaren Groupware-Systeme recherchiert.
Nachdem das Groupware-System ausgewählt wurde, wurde es auf einer Cloud-Instanz installiert und konfiguriert.
Zuletzt wurden automatisierte End-to-End-Tests für das Groupware-System implementiert und durchgeführt.

Das Ziel des Studienprojekts wurde insofern erreicht, als dass ein erstes Groupware-System installiert und getestet wurde.
Das untersuchte Groupware-System, EGroupware, erfüllt dabei die funktionalen Anforderungen an ein Groupware-System, wie E-Mails, Kalender mit Terminplanung und Kontakte.
Auch die nicht-funktionale Anforderung der Eigenverwaltbarkeit wurde erfüllt, da die Installation und Konfiguration des Systems auf einer Cloud-Instanz durchgeführt werden konnte.

Ein Faktor, bei dem bei EGroupware noch Zweifel aufwirft, ist die Automatisierbarkeit des Systems.
Da bei Tests mit Playwright zum Erstellen und Löschen von Nutzern Asynchronitäten in den angezeigten Nutzeraccounts auftraten, ist in Frage zu stellen, ob das System für ausführliche Automatisierung geeignet ist.
Daher kann ohne weitere ausführliche Tests noch nicht eindeutig gesagt werden, ob EGroupware die Anforderungen der Hochschule Esslingen am besten erfüllt.


\section{Ausblick}

In diesem Studienprojekt wurde die Basis der Suche nach einem neuen Groupware-System für die Hochschule Esslingen gelegt.
Dabei wurden die Anforderungen an das Groupware-System definiert und ein erstes Groupware-System installiert und getestet.

Im nächsten Schritt soll, möglicherweise in weiteren Studienprojekten, die Recherche nach einem Groupware-System mit der Installation und dem Testen weiterer Groupware-Systeme fortgesetzt werden.
Alle der in diesem Studienprojekt betrachteten Groupware-Systeme erfüllen die grundsätzlichen Anforderungen der Hochschule an ein neues Groupware-System.
Daher könnte es sinnvoll sein, in Zukunft noch Kolab, Horde und Sogo, die bereits von deutlich mehr Hochschulen genutzt werden als EGroupware, als mögliche Lösungen für die Hochschule Esslingen zu betrachten.

Falls EGroupware weiterhin als Lösung für die Hochschule Esslingen in Betracht gezogen wird, sollte die Automatisierbarkeit des Systems genauer untersucht werden.
Beispielsweise sollte die Robustheit des Systems beim Erstellen vieler Nutzer auf einmal getestet werden, um sicherzustellen, dass das System die Anforderungen der Hochschule Esslingen erfüllt.


