% % Neue Befehle
\newcommand{\HRule}[2]{\noindent\rule[#1]{\linewidth}{#2}} % Horiz. Linie
\newcommand{\vlinespace}[1]{\vspace*{#1\baselineskip}} % Abstand
\newcommand{\titleemph}[1]{\textbf{#1}} % Hervorheben

\begin{titlepage}
 \sffamily % Titelseite in seriefenloser Schrift
      % Logo Hochschule Esslingen
      \includegraphics[width=4cm]{images/it-d_logo.png}
      \hfill \includegraphics[width=5cm]{images/hslogo_small.png}
      \HRule{13pt}{1pt} 
   \centering
      \Large
      \vlinespace{3}\\
      Bachelorarbeit\\
      \huge
      Interaktive Visualisierung von Software-Requirements mit Augmented Reality: Eine Analyse der Usability und Effektivität\\
%
      \Large
      \vlinespace{2}
          im Studiengang Softwaretechnik und Medieninformatik (SWB)\\
          der Fakultät Informationstechnik\\
%      
      Sommersemester 2024\\
%     
      \vlinespace{2}
      Kyle Mezger\\
      Matrikelnummer: 765838
%
   \vfill
   \raggedright
%   
   \large
  \titleemph{Zeitraum:} 01.03.2024 bis 31.08.2024 \\ % Nur bei Abschluss-Arbeiten
   \titleemph{Erstprüfer:} Prof. Dr. -Ing. Andreas Rößler \\
   \titleemph{Zweitprüfer:} Prof. Dr. rer. nat. Dieter Morgenroth \\

   % Folgenden Abschnitt nur bei Industrie-Arbeiten darstellen
   \vlinespace{1}
   \HRule{13pt}{1pt} \\
   \titleemph{Firma:} IT Designers Gruppe 
   \hfill \titleemph{Betreuer:} Stefan Kaufmann

\end{titlepage}
