\chapter{Kurzfassung}

Die vorliegende Bachelorarbeit befasst sich mit der Visualisierung von Requirements mithilfe von Augmented Reality.

Unter Requirements werden Anforderungen verstanden, die vor der Entwicklung eines Systems an dieses gestellt werden.
Sie stellen einen essenziellen Bestandteil der Entwicklung von Produkten und Systemen dar, da sie die Basis für die Entwicklung bilden.
Die Visualisierung von Requirements zielt darauf ab, im Verlauf des Requirements-Engineering den Überblick zu bewahren und die Requirements besser zu verstehen.
Dazu werden zwei verschiedene Interaktionskonzepte ausgearbeitet und jeweils in einem WebXR-Prototyp mit Babylon.js umgesetzt.

Das erste Konzept umfasst die Visualisierung von Requirements in Kombination mit einem 3D-Modell des zu entwickelnden Produkts.
In diesem Zusammenhang wird das 3D-Modell, für welches in dieser Arbeit beispielhaft ein Fahrzeug verwendet wird, in einer Augmented-Reality-Umgebung dargestellt.
Das 3D-Modell kann dann in einer Animation in seine Einzelteile zerlegt werden, um die Requirements als Panels an den einzelnen Bauteilen zu visualisieren. 

Der Prototyp demonstriert den Mehrwert des Interaktionskonzepts durch AR, welcher sich aus der direkten räumlichen Verknüpfung von Requirements und Bauteilen ergibt.
Des Weiteren können in dem Interaktionskonzept theoretisch noch zahlreiche Interaktionsmöglichkeiten des Nutzers mit den Requirements und dem 3D-Modell implementiert werden.
Allerdings ist die Implementierung des Konzepts in Anwendungen aufwendig, da viel Handarbeit erforderlich ist.
Ohne weitere Automatisierung der Implementierung ist daher die wirtschaftliche Effektivität des Konzepts fraglich.

Das zweite Interaktionskonzept basiert auf einer Visualisierung von Requirements in Wolken aus Textpanels, die in der Augmented-Reality-Umgebung schweben.
Jedes Panel beinhaltet dabei ein Requirement.
Die Wolken fungieren als Cluster von Requirements, die thematisch zusammengehören.
Durch Anklicken eines Panels sollen die zugehörigen Requirements angezeigt werden.

Dieses Konzept führt schon bei kleinen Wolken zu Überlappungen der Requirementpanels.
Daher ist die Weiterentwicklung des Konzepts in 3D, bzw. AR, nicht sinnvoll.
Allerdings wäre es denkbar, das Interaktionskonzept in 2D als zusätzliche Ansicht in klassischen Requirement-Engineering-Tools wie Jira zu integrieren.
Aus diesem Grund wurde die Implementierung des Konzepts als zweidimensionale Anwendung im Rahmen dieser Bachelorarbeit als Mockup ausgearbeitet und untersucht.

