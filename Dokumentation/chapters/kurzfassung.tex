\chapter{Kurzfassung}

Diese Arbeit beschäftigt sich mit der Recherche, Evaluierung, Implementierung und dem automatisierten End-to-End-Testing von Groupware-Systemen für die Hochschule Esslingen.
Besonderer Wert wird auf den Einsatz von Open Source Software gelegt, die selbst administrierbar sein soll und möglichst von einer deutschen Firma entwickelt wurde.
\\
\\
Groupwaresysteme sind Softwareanwendungen, die die Zusammenarbeit und Organisation von Arbeitsgruppen unterstützen.
Sie bieten Funktionen wie das Anlegen von gemeinsamen Terminen, das Erstellen von Projektplänen sowie das Versenden und Empfangen von E-Mails.
Auch wird der Prozess der Recherche und Bewertung verschiedener Groupware-Systeme erläutert.
Es werden vier verschiedene Groupware-Systeme beschrieben und die Kriterien erläutert, anhand derer die Systeme analysiert und bewertet wurden.
\\
\\
Im Anschluss wird das ausgewählte Groupware-System EGroupware vorgestellt und dessen Installation und Konfiguration beschrieben.
In diesem Zusammenhang werden auch die verschiedenen Funktionen von EGroupware erläutert und es wird beschrieben, wie diese Funktionen genutzt werden können.
Danach wird auf den Aufbau der Testumgebung und die Implementierung der Tests eingegangen und die Implementierung verschiedener automatisierter End-to-End Tests für EGroupware beschrieben.
Dabei werden wichtige Konzepte für die Implementierung und die abgedeckten Bereiche der Tests näher erläutert.
\\
\\
Abschließend wird ein Fazit über die Ergebnisse der Arbeit gezogen und ein Ausblick auf mögliche zukünftige Erweiterungen, wie beispielsweise zukünftige Studienarbeiten die andere der Groupwaresysteme testen, gegeben.