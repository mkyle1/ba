\chapter{Kurzfassung}

Diese Arbeit befasst sich mit der Recherche, Bewertung, Implementierung und dem automatisierten Testen von Groupwaresystemen.
Dabei liegt eine besonderer Fokus auf der Verwendung von Open Source Software von deutschen Firmen.

Groupwaresysteme sind Softwareanwendungen, die die Zusammenarbeit und Organisation von Arbeitsgruppen unterstützen.
Dabei bieten sie Funktionen wie beispielsweise das Anlegen von Terminen, das Erstellen von Projektplänen oder das versenden und empfangen von Emails.


Dabei wurden von der Hochschule Esslingen folgende Vorgaben für das Groupwaresystem gemacht:

Auch wird der Prozess der Recherche und Bewertung verschiedener Groupwaresysteme erläutert.
Dabei wird auf 3 verschiedene Groupwaresysteme eingegangen und beschrieben nach welchen Kriterien diese analysiert und bewertet wurden.
