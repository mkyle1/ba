\chapter{Kurzfassung}

Diese Arbeit beschäftigt sich mit der Recherche, Evaluierung, Implementierung und dem automatisierten End-to-End-Testing von Groupware-Systemen für die Hochschule Esslingen.
Besonderer Wert wird auf den Einsatz von Open-Source-Software gelegt, die selbst administrierbar sein soll und möglichst von einer deutschen Firma entwickelt wurde.

Groupware-Systeme sind Softwareanwendungen, die die Zusammenarbeit und Organisation von Arbeitsgruppen unterstützen.
Sie bieten Funktionen wie das Anlegen von gemeinsamen Terminen, das Erstellen von Projektplänen sowie das Versenden und Empfangen von E-Mails.
Auch wird der Prozess der Recherche und Bewertung verschiedener Groupware-Systeme erläutert.

Im Umfang der Recherche werden vier verschiedene Groupware-Systeme und deren Funktionen beschrieben und die Kriterien erläutert, anhand derer die Systeme analysiert und bewertet wurden.
Dabei wird die Entscheidung für das Groupware-System EGroupware begründet und die Installation und Konfiguration des Systems auf einer Cloud-Instanz beschrieben.

Anschließend geht die Studienarbeit auf den Aufbau der Testumgebung und die Implementierung der Tests ein, wobei auch die Arbeitsweise mit der Testing-Bibliothek Playwright und dessen Tools beschrieben wird.
Zudem wird genauer auf die technische Implementierung verschiedener automatisierter End-to-End-Tests für EGroupware eingegangen, die alltägliche Nutzerszenarien abdecken sollen.
Dabei werden wichtige Konzepte für die Implementierung und die abgedeckten Bereiche der Tests näher erläutert.

Abschließend wird ein Fazit über die Ergebnisse der Studienarbeit und die Eignung von EGroupware als Groupware-System für die Hochschule Esslingen gezogen.
Dabei wird auch ein Ausblick auf mögliche nächste Schritte in der Suche nach einer Groupware-Lösung für die Hochschule Esslingen gegeben, wie beispielsweise die Untersuchung weiterer Groupware-Systeme aus der Recherche dieser Arbeit in weiteren Studienarbeiten.