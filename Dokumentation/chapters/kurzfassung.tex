\chapter{Kurzfassung}

Die Bachelorarbeit beschäftigt sich mit der Visualisierung von Requirements mithilfe von Augmented Reality.

Requirements sind dabei Anforderungen, die vor der Entwicklung eines Systems an dieses gestellt werden.
Sie sind ein essenzieller Bestandteil der Entwicklung von Produkten und Systemen, da sie die Basis für die Entwicklung bilden.
Die Visualisierung von Requirements soll dabei helfen, im fortlaufenden Prozess des Requirements-Engineering den Überblick zu behalten und die Anforderungen besser zu verstehen.
Dafür werden 2 verschiedene Interaktionskonzepte ausgearbeitet und in jeweils einem WebXR-Prototypen mit Babylon.js umgesetzt.

Das erste Konzept ist eine Visualisierung von Requirements in Kombination mit einem 3D-Modell des zu entwickelnden Produkts.
Dabei wird das 3D-Modell, für welches in dieser Arbeit beispielhaft ein Auto verwendet wird, in einer Augmented-Reality-Umgebung dargestellt.
Dieses Modell kann in einer Animation in seine Einzelteile zerlegt werden, um dann die Anforderungen als Panels an den einzelnen Bauteilen zu visualisieren.

Das zweite Konzept ist eine Visualisierung von Requirements in Wolken aus Textpanels, die in der Augmented-Reality-Umgebung schweben.
Jedes Panel beinhaltet dabei ein Requirement.
Die Wolken sollen dann Cluster von Anforderungen sein, die thematisch zusammengehören.
Die Anforderungen sollen dann angeklickt werden können, um zugehörige Anforderungen anzuzeigen.


Im Rahmen der Bachelorarbeit wird ein Prototyp entwickelt, der die beiden Konzepte umsetzt.
Daraufhin werden die Prototypen anhand ihrer Usability und der Effektivität des Konzepts in AR analysiert.