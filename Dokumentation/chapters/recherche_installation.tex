\chapter{Installation des Groupwaresystems}

In diesem Kapitel wird genauer auf das final gewählte Groupware-System EGroupware sowie dessen Installation und Deployment eingegangen.

\section{Alle Kandidaten}

Zu Beginn der Studienarbeit wurden anhand der gegebenen Kriterien mehrere Kandidaten für das Groupware-System recherchiert, um einen Überblick über die verfügbaren Möglichkeiten zu erhalten.
Dabei wurden 4 erfolgsversprechende Kandidaten gefunden, über die im Folgenden kurz berichtet wird.
Da die Wahl schlussendlich auf EGroupware fiel, wird im Anschluss genauer auf dieses System eingegangen.

\subsection{Kolab}

Das Groupwaresystem Kolab wird von der schweizerischen Firma Aphelia IT AG entwickelt und ist als Open-Source Produkt gratis verfügbar und bietet die folgenden Features:
\begin{itemize}
    \item E-Mail
    \item Kalender
    \item Kontakte
    \item Online-File-Server
    \item Aufgabenmanagement
    \item Notizen
    \item Sprach- und Videoanrufe
\end{itemize}
\autocite{kolab}

\subsection{Horde}

Horde wird von einem gleichnamigen amerikanischen Unternehmen entwickelt und ist wie die anderen Systeme Open-Source und gratis verfügbar. Es bietet dabei die folgenden Funktionalitäten:
\begin{itemize}
    \item Kalender
    \item Kontakte
    \item E-Mail
    \item Terminmanagement
    \item Projektmanagement
    \item Dokumentenmanagement
    \item Online-File-Server
\end{itemize}
\autocite{horde}
\\
Relevant für die Auswahl könnte auch sein, dass Horde schon von einigen anderen Universitäten und Hochschulen, wie beispielsweise der Universität Tübingen und Universität Paderborn, verwendet wird.

\subsection{Sogo}

Sogo ist ein Open-Source Groupwaresystem welches von der französischen Firma Alinto entwickelt wird.
Das Systemist frei verfügbar und bietet die folgenden Features:

\begin{itemize}
    \item E-Mail
    \item Kalender
    \item Kontakte
    \item Erinnerungen
    \item 2-Faktor-Authentifizierung
    \item Raum Reservationen
\end{itemize}
\autocite{sogo}
\\
Ähnlich wie Horde wird auch Sogo von einigen Universitäten und Hochschulen verwendet, wie beispielsweise der Universität Koblenz und der Universität Ulm.

\subsection{EGroupware}

EGroupware ist ein Open-Source Groupwaresystem welches von einer deutschen Firma entwickelt wird.
Das System läuft auf einem Host Server und kann dann über einen Webbrowser genutzt werden.
Daher kann das System auch auf Smartphones und Tablets genutzt werden.
Zudem bietet das System Integrationsmöglichkeiten für LDAP Usermanagement sowie eigene Mail Server.

Als Funktionalitäten bietet EGroupware:

\begin{itemize}
    \item Kalender
    \item Kontakte
    \item E-Mail
    \item Terminmanagement
    \item Projektmanagement
    \item Dokumentenmanagement
    \item Online-File-Server
\end{itemize}


\section{Entscheidung für EGroupware}

Bei der Recherche der verschiedenen Groupwaresysteme wurde klar, dass alle der untersuchten Systeme die grundlegenden gewünschten Funktionalitäten bieten und daher grundsätzlich alle für die Hochschule Esslingen geeignet sind.
Durch diesen Umstand konnte die Entscheidung nicht rein aufgrund der Funktionalitäten der Systeme getroffen werden, da sich keines der Systeme in diesem Punkt stark von den anderen abhebt.
Somit fiel final die Entscheidung auf EGroupware, da es von einer deutschen Firma entwickelt wird und umfangreiche Funktionen bietet, die für die Hochschule Esslingen relevant sein könnten.




\section{Installation auf BWCloud}

Für die Installation der EGroupware Software stellt die EGroupware GmbH eine Installationsanleitung zu Verfügung.
Darin wird auf einer Linux Instanz ein Docker Repository hinzugefügt und anschließend die EGroupware Software installiert.
Alle dafür benötigten Konsolenbefehle waren in der Installationsanleitung angegeben. \autocite{egroupware-installation}

Für das Hosting der EGroupware Software wurde sowohl WSL2 (Windows-Subsystem für Linux) als auch eine BWCloud Instanz getestet.
Die reine Installation der Software war auf beiden Systemen auf einer Ubuntu22.04 Instanz möglich.
Im Laufe der Installation wurde jedoch klar, dass die Netzwerkkonfiguration bei BWCloud einfacher zu handhaben ist als bei WSL2.
Daher wurde die finale Installation und Konfiguration auf einer BWCloud Instanz durchgeführt.
Dafür mussten einige Ports freigegeben werden, damit die EGroupware Software von außerhalb der BWCloud Instanz erreichbar ist:
\begin{itemize}
    \item Port 80 für HTTP
    \item Port 443 für HTTPS
    \item Port 8080 für HTML
\end{itemize}

Für den ersten Zugriff auf die Software für beispielsweise das Erstellen von Nutzer Accounts wird bei der Installation automatisch ein Administrator Account angelegt, dessen Zugangsdaten im Log der Installation zu finden sind.