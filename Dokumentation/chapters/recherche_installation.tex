\chapter{Installation des Groupwaresystems}

In diesem Kapitel wird genauer auf das final gewählte Groupware-System EGroupware sowie dessen Installation und Deployment eingegangen.

\section{Alle Kandidaten}

Zu Beginn der Studienarbeit wurden anhand der gegebenen Kriterien mehrere Kandidaten für das Groupware-System recherchiert, um einen Überblick über die verfügbaren Möglichkeiten zu erhalten.

\subsection{Kolab}

\autocite{kolab}

\subsection{Horde}

\autocite{horde}

\subsection{Sogo}

\autocite{sogo}

\subsection{EGroupware}

EGroupware ist ein Open-Source Groupwaresystem welches von einer deutschen Firma entwickelt wird.
Das System läuft auf einem Host Server und kann dann über einen Webbrowser genutzt werden.
Daher kann das System auch auf Smartphones und Tablets genutzt werden.
Zudem bietet das System Integrationsmöglichkeiten für LDAP Usermanagement sowie eigene Mail Server.

Als Funktionalitäten bietet EGroupware:

\begin{itemize}
    \item Kalender
    \item Kontakte
    \item E-Mail
    \item Terminmanagement
    \item Projektmanagement
    \item Dokumentenmanagement
    \item Online-File-Server
\end{itemize}

Damit bietet das System mehr als alle gewünschten Funktionalitäten und ist daher für die Hochschule Esslingen geeignet.
\section{EGroupware}



\section{Installation auf BWCloud}