\chapter{Installation des Groupwaresystems}

In diesem Kapitel wird genauer auf das final gewählte Groupware-System EGroupware sowie dessen Installation und Deployment eingegangen.

\section{Alle Kandidaten}

Zu Beginn der Studienarbeit wurden anhand der gegebenen Kriterien mehrere Kandidaten für das Groupware-System recherchiert, um einen Überblick über die verfügbaren Möglichkeiten zu erhalten.
Dabei wurden 4 erfolgsversprechende Kandidaten gefunden, über die im Folgenden kurz berichtet wird.
Da die Wahl schlussendlich auf EGroupware fiel, wird im Anschluss genauer auf dieses System eingegangen.

\subsection{Kolab}

Das Groupwaresystem Kolab wird von der schweizerischen Firma Aphelia IT AG entwickelt und ist als Open-Source Produkt gratis verfügbar und bietet die folgenden Features:

\begin{itemize}
    \item E-Mail
    \item Kalender
    \item Kontakte
    \item Online-File-Server
    \item Aufgabenmanagement
    \item Notizen
    \item Sprach- und Videoanrufe
\end{itemize}

\autocite{kolab}

\subsection{Horde}

Horde wird von einem gleichnamigen amerikanischen Unternehmen entwickelt und ist wie die anderen Systeme Open-Source und gratis verfügbar.

\autocite{horde}

\subsection{Sogo}

Sogo ist ein Open-Source Groupwaresystem welches von der französischen Firma Alinto entwickelt wird.
Das System ist frei verfügbar und bietet die folgenden Features:

\begin{itemize}
    \item E-Mail
    \item Kalender
    \item Kontakte
    \item Erinnerungen
    \item 2-Faktor-Authentifizierung
    \item Raum Reservationen
\end{itemize}

\autocite{sogo}

\subsection{EGroupware}

EGroupware ist ein Open-Source Groupwaresystem welches von einer deutschen Firma entwickelt wird.
Das System läuft auf einem Host Server und kann dann über einen Webbrowser genutzt werden.
Daher kann das System auch auf Smartphones und Tablets genutzt werden.
Zudem bietet das System Integrationsmöglichkeiten für LDAP Usermanagement sowie eigene Mail Server.

Als Funktionalitäten bietet EGroupware:

\begin{itemize}
    \item Kalender
    \item Kontakte
    \item E-Mail
    \item Terminmanagement
    \item Projektmanagement
    \item Dokumentenmanagement
    \item Online-File-Server
\end{itemize}

Damit bietet das System mehr als alle gewünschten Funktionalitäten und ist daher für die Hochschule Esslingen geeignet.


\section{Entscheidung für EGroupware}

Bei der Recherche der verschiedenen Groupwaresysteme wurde klar, dass alle der untersuchten Systeme die grundlegenden gewünschten Funktionalitäten bieten und daher grundsätzlich alle für die Hochschule Esslingen geeignet sind.
Durch diesen Umstand kann die Entscheidung nicht rein aufgrund der Funktionalitäten der Systeme getroffen werden, da sich keines der Systeme in diesem Punkt stark von den anderen abhebt.
Daher wurde die Entscheidung basierend auf dem Land der Entwicklungsfirma getroffen, da dies auch ein Kriterium, wenn auch kein striktes Ausschlusskriterium, bei der Suche nach den Groupwaresystemen war.

Somit fiel die Entscheidung auf EGroupware, da es das einzige System der Auswahl ist, dessen Entwicklungsfirma in Deutschland sitzt.


\section{Installation auf BWCloud}

